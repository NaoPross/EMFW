% !TeX program = xelatex
% !TeX encoding = utf8
% !TeX root = Vorlage.tex
% vim: ts=2 sw=2 et tw=78:

\documentclass[margin=tiny]{hsrzf}

%%%%%%%%%%%%%%%%%%%%%%%%%%%%%%%%%%%%%%%%%%%%%%%%%%%
% Packages

%% The OST Student's package
\usepackage{oststud}

%% Language configuration
\usepackage{polyglossia}
\setdefaultlanguage{english}

%% License configuration
\usepackage[
  type={CC},
  modifier={by-nc-sa},
  version={4.0},
]{doclicense}

%%%%%%%%%%%%%%%%%%%%%%%%%%%%%%%%%%%%%%%%%%%%%%%%%%%
% Metadata

\course{Electrical Engineering and Information Technology}
\module{Electromagnetic Fields and Waves}
\semester{Spring Semester 2023}

\authoremail{npross@ethz.ch}
\author{\textsl{Naoki Sean Pross} -- \texttt{\theauthoremail}}

% did someone help you with this work?
\contributors{
  % I created this template, does that count?
  Naoki Pross
  % do not forget to add yourself!
}

\title{\themodule}
\date{\thesemester}

%%%%%%%%%%%%%%%%%%%%%%%%%%%%%%%%%%%%%%%%%%%%%%%%%%%
% Macros

\newcommand{\cvec}[1]{\underline{\vec{#1}}}

%%%%%%%%%%%%%%%%%%%%%%%%%%%%%%%%%%%%%%%%%%%%%%%%%%%
% Document

\begin{document}

% use roman numberals for introductiory pages
\pagenumbering{roman}

\maketitle

% \begin{abstract}
% \end{abstract}

% show the names of the people who contributed to this document.
% \section*{Contributors}
% \thecontributors

\section*{Lizenz}
\doclicenseThis

\tableofcontents

% actual content
\clearpage
\setcounter{page}{1}
\pagenumbering{arabic}

% I like to use 2 columns, you can remove this line to have one column.
\twocolumn

\section{Maxwell's Equations}

\subsection{Time Domain}

Maxwell's equations in matter in their integral form are:
\begin{align*}
  &\textit{Gauss} &
    \oint_{\partial V} \vec{D} \dotp d\vec{s}
    &= \int_V \rho_0 dv, \\
  &\textit{Ampère} &
    \oint_{\partial A} \vec{H} \dotp d\vec{s}
    &= \int \left[ \vec{J}_0 + \partial_t \vec{D} \right] d\vec{s}, \\
  &\textit{Faraday} &
    \oint_{\partial A} \vec{E} \dotp d\vec{s} 
    &= -\partial_t \int_A \vec{B} \dotp d\vec{s}, \\
  &\textit{Gauss} &
    \int_{\partial V} \vec{B} \dotp d\vec{s}
    &= 0.
\end{align*}
Where the secondary sources are contained in the constitutive relations for
the \emph{polarization} and \emph{magnetization} of materials:
\begin{align*}
  \vec{D} &= \varepsilon \vec{E} + \vec{P}, &
  \vec{B} &= \mu_0 \left[ \vec{H} + \vec{M} \right].
\end{align*}

In their differential form Maxwell's equatiosn are
\begin{align*}
  &\textit{Gauss} &
    \div \vec{D} &= \rho_0, &
  &\textit{Ampère} &
    \curl \vec{H} &= \partial_t \vec{D} + \vec{J}_0, \\
  &\textit{Faraday} &
    \curl \vec{E} &= - \partial_t \vec{B}, &
  &\textit{Gauss} &
    \div \vec{B} &= 0,
\end{align*}

Maxwell's equations implicitly contain the continuity equation for the
conservation of charge. Taking the divergence of Ampère's law and using the
fact that $\div(\curl \vec{H} = 0)$, and that $\div \vec{D} = \rho_0$:
\[
  0 = \partial_t \rho_0 + \div \vec{J}.
\]

\subsection{Spectral Representation}

Applying the Fourier transform to Maxwell's equation we get
\begin{align*}
  &\textit{Gauss} &
    \div \hat{\vec{D}} &= \rho_0, &
  &\textit{Ampère} &
    \curl \hat{\vec{H}} &= -i\omega \hat{\vec{D}} + \hat{\vec{J}}_0, \\
  &\textit{Faraday} &
    \curl \hat{\vec{E}} &= i\omega \hat{\vec{B}}, &
  &\textit{Gauss} &
    \div \hat{\vec{B}} &= 0.
\end{align*}
For monochromatic fields we can replace each quantity with its complex
auxiliary field $\hat{\vec{Q}}(\vec{r}, \omega) \leadsto \cvec{Q}(\vec{r})$.

\section{Electromagnetic Waves}

\subsection{The Wave Equation}

Combining Ampère and Faraday's laws by expressing the fluxes ($\vec{D}$,
$\vec{B}$) in term of field strenghts ($\vec{E}$, $\vec{H}$) and taking their
curl results in the wave equations
\begin{subequations}
  \begin{align*}
    \curl \curl \vec{E} + \frac{1}{c^2} \partial^2_t \vec{E}
      &= - \mu_0 \partial_t \left(
          \vec{J}_0 + \partial_t \vec{P} + \curl \vec{M}
        \right), \\
    \curl \curl \vec{H} + \frac{1}{c^2} \partial^2_t \vec{H}
      &= \curl \vec{J}_0 + \curl \partial_t \vec{P}
          - \frac{1}{c^2} \partial_t^2 \vec{M}.
  \end{align*}
\end{subequations}

In free space since $\div \vec{E} = 0$, $\vec{J} = \vec{0}$ and $\vec{M} =
\vec{0}$, then using a vector calculus identity we get
\begin{align*}
  \vlaplacian \vec{E} - \frac{1}{c^2} \partial^2_t \vec{E} &= \vec{0}, &
  \vlaplacian \vec{H} - \frac{1}{c^2} \partial^2_t \vec{H} &= \vec{0}. &
\end{align*}

\subsection{Monochromatic Fields and Waves}

These wave equations are solved using the separation Ansatz, $\vec{E}(\vec{r},
t) = \vec{R}(\vec{r}) T(t)$. Solving the ODE for the time component we get
$T(t) = \Re{c_\omega e^{-i\omega t}}$, with $c_\omega \in \mathbb{C}$. We
define $\cvec{E}(\vec{r}) = c_\omega \vec{R}(\vec{r})$ and call it the
\emph{complex auxiliary field}, and the physical solution is $\vec{E}(\vec{r},
t) = \Re{\cvec{E}(\vec{r})e^{-i\omega t}}$. Since this depends on a specific
$\omega$ we call this a \emph{monochromatic} field.

The simplest monochromatic field is a plane wave, from the Ansatz
$\cvec{E}(\vec{r}) = \cvec{E}_0 e^{\pm i\vec{k} \dotp \vec{r}}$ which when
plugged in the wave equations becomes the Helmholtz equation
\[
  \vlaplacian \cvec{E}(\vec{r}) + k^2 \cvec{E}(\vec{r}) = 0
\]
which gives the dispersion relation
\[
  k^2 = \vec{k} \dotp \vec{k} = k_x^2 + k_y^2 + k_z^2 =
  \frac{\omega^2}{c^2}.
\]
Then because $\div \vec{E} = \Re{\vec{k} \dotp \cvec{E}_0 e^{i(\vec{k}
\dotp\vec{r} - \omega t)}} \iff \vec{k} \dotp \cvec{E}_0 = \vec{0}$
(orthogonal). The corresponding magnetic field is given by Faraday's law that
becomes $\curl \cvec{E} = i\vec{k} \crossp \cvec{E}_0 e^{i\vec{k}\dotp
\vec{r}} = i\omega \mu_0 \cvec{H}$, which means that
\[
  \cvec{H}_0 = \frac{1}{\omega \mu_0} \vec{k} \crossp \cvec{E}_0 =
  \frac{\vec{k} \crossp \cvec{E}_0}{Z_0},
  \quad \text{and} \quad
  \cvec{H}_0 ~\bot~ \cvec{E}_0.
\]
The temporal spectrum (Fourier transform) of a monochromatic wave is two dirac deltas
\[
  \hat{\vec{E}}(\vec{r}, \omega') = \frac{1}{2} \left[
    \cvec{E}(\vec{r}) \delta(\omega' - \omega)\
    + \cvec{E}^*(\vec{r}) \delta(\omega' + \omega) \right].
\]

\subsection{Evanescent Waves}

Because $k^2 = \omega^2 / c^2$, then $k_z = \sqrt{k^2 - k_x^2 - k_y^2}$, and
if $k_x^2 + k_y^2 > \omega^2 / c^2$ then $k_z$ becomes imaginary and results
in an exponential decaying wave, also called \emph{evanescent} wave
\[
  \vec{E}(\vec{r}, t) = \Re{\cvec{E}_0 e^{\pm i(k_x x + k_y y - i\omega t)}}
    e^{\pm|k_z| z}.
\]
These waves are a form of stored energy (reactive power) and can be found near
sources or at material interfaces.

\subsection{General Solutions}

We can construct general solutions to the wave equation by combining plane
(and implicitly evanescent) waves, infinitely many of them,
\[
  \vec{E}(\vec{r}, t) = \Re{\iint \hat{\cvec{E}}_0(\vec{k}, \omega)
    e^{i(\vec{k}\cdot \vec{r} - \omega t)} d\omega d\vec{k}}.
\]
Inside the real operator is a 4D Fourier transform of the \emph{complex
amplitude density} $\hat{\cvec{E}}_0(\vec{k}, \omega)$. For finite bandwidth
the solution is given in term of the \emph{temporal spectrum}
$\vec{E}(\vec{r}, \omega)$:
\[
  \vec{E}(\vec{r}, t) = \Re{
    \int_\mathbb{R} \hat{\vec{E}}(\vec{r}, \omega) e^{-i\omega t} d\omega
  } = \int_\mathbb{R} \hat{\vec{E}}(\vec{r}, \omega) \, d\omega,
\]
where to remove $\Re{\cdot}$ we impose the symmetry $\hat{\vec{E}}(\vec{r},
-\omega) = \hat{\vec{E}}^*(\vec{r}, \omega)$. The inverse transform is then
\[
  \hat{\vec{E}}(\vec{r}, \omega) = \frac{1}{2\pi} \int_\mathbb{R}
  \vec{E}(\vec{r}, t) e^{i\omega t} dt.
\]

\subsection{Polarization}

A plane wave is \emph{linearly} polarized if $\cvec{E}_0 = \vec{E}_0
e^{i\phi}$. Ciricular polarization is when there is a phase difference of $\pm
\pi / 2 \equiv \pm i$ in two components:
\begin{align*}
  z \text{--Left}\quad
    & \cvec{E}(\vec{r}) = E_0 e^{i\phi} \begin{bmatrix}
      0 & 1 & -i
    \end{bmatrix}^\mathsf{T} e^{ikz}, \\
  z \text{--Right}\quad
    & \cvec{E}(\vec{r}) = E_0 e^{i\phi} \begin{bmatrix}
      0 & 1 & i
    \end{bmatrix}^\mathsf{T} e^{ikz}.
\end{align*}
Then $\Re{\cvec{E}(\vec{r})e^{-i\omega t}}$ will contain sines and cosines.
\emph{Elliptic} polarization is when the two out of phase components have
different amplitudes.

\subsection{Formulation with Angle of Incidence}

\section{Waves in Matter and Constitutive Relations}

\subsection{Linear Materials}

We consider \emph{linear piecewise isotropic} materials, that is $\vec{D}$ is
linear in $\vec{E}$ and $\vec{B}$ in $\vec{H}$. The most general linear
relationship is given when the material is both \emph{spatially dispersive}
and \emph{temporally dispersive}
\[
  \vec{D}(\vec{r}, t) = \varepsilon_0 \iint_{\mathbb{R}^4} \tilde{\varepsilon}(\vec{r} -
  \vec{r}', t - t') \vec{E}(\vec{r}', t') ~d\vec{r}' dt',
\]
that is, the local field depends on material in the neighborhood and its past.
To avoid the covolution we use the Fourier transform to get
$\hat{\vec{D}}(\vec{k}, \omega) = \varepsilon_0 \varepsilon(\vec{k}, \omega)
\hat{\vec{E}}(\vec{k}, \omega)$. Spatial dispersion can be ignored, but
temporal dispersion cannot, hence we work with the \emph{dielectric function}
$\varepsilon(\vec{r}, \omega)$. In the same way for $\vec{H}$ and $\vec{B}$
there is the magnetic permeability $\mu(\vec{r}, \omega)$. For time
\emph{independent} fields we can write
\[
  \cvec{D}(\vec{r}) = \varepsilon_0\varepsilon(\vec{r}, \omega)
    \cvec{E}(\vec{r}), \qquad
  \cvec{B}(\vec{r}) = \mu_0\mu(\vec{r}, \omega) \cvec{H}(\vec{r}).
\]
The same relations are also written using the \emph{susceptibilities}
$\chi_e = \varepsilon -1$, $\chi_m = \mu - 1$:
\[
  \cvec{P}(\vec{r}) = \epsilon_0\chi_e(\vec{r}, \omega)\cvec{E}(\vec{r}),
  \qquad
  \cvec{M}(\vec{r}) = \chi_m(\vec{r}, \omega) \cvec{H}(\vec{r}).
\]
The wave equations in linear media are
\begin{align*}
  \curl \mu(\vec{r}, \omega)^{-1} \curl \cvec{E}(\vec{r})
    - k_0^2 \varepsilon(\vec{r}, \omega) \cvec{E}(\vec{r})
    &= i\omega \mu_0 \cvec{J}_0(\vec{r}) \\
  \curl \varepsilon(\vec{r}, \omega)^{-1} \curl \cvec{H}(\vec{r})
    - k_0^2 \mu(\vec{r}, \omega) \cvec{H}(\vec{r})
    &= \curl \varepsilon(\vec{r}, \omega)^{-1} \cvec{J}_0(\vec{r}),
\end{align*}
where $k_0 = \omega / c$ and the quantity $n = \sqrt{\varepsilon \mu}$ is the
\emph{refraction index}. In absence of sources and isotropic $\mu(\omega)$ we
obtain the Helmholz equation with $k^2 = k_0^2n^2$:
$\vlaplacian\cvec{E}(\vec{r}) + k_0^2n^2\cvec{E}(\vec{r}) = \vec{0}$.

\subsection{Conductivity and Conductors}

Conductivity $\sigma$ relates $\cvec{J}_\text{c}(\vec{r}) = \sigma(\vec{r}, \omega)
\cvec{E}(\vec{r})$. Inserting the linear relation $\cvec{D} =
\varepsilon_0\varepsilon \cvec{E}$ in Ampère's law and letting $\varepsilon =
\varepsilon' + i\varepsilon''$
be complex results in
\[
  \sigma(\vec{r}, \omega) = \omega \varepsilon_0\varepsilon''(\vec{r},
  \omega).
\]

Inside of ideal conductors there are no free charges. For non ideal conductors
the charge inside dissipates within a time $T_\rho = \varepsilon_0\varepsilon
/ \sigma$.

\subsection{Skin Effect}

For a good conduct

\subsection{Boundary Conditions}
\subsection{Polarization}

\section{Conservation Laws}

\appendix

\section{Fourier Transform}

\section{Vector Analysis}

\end{document}
